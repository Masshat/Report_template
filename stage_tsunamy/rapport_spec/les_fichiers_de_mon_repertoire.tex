\chapter{Les fichiers de mon r�pertoire au laboratoire}

Vous devez faire � l'attention de votre encadrant, une annexe d�crivant de mani�re d�taill�e des fichiers produits pendant votre stage et un mode d'emploi d�taill� qui permet d'utiliser votre r�alisation.
   \begin{itemize}
   \item
La documentation que vous avez utilis�e : o� est-elle ? O� est le fichier .bib qui regroupe les r�f�rences ?
   \item
Votre algorithme : est-il bien d�crit dans le rapport ? o� sont les calculs interm�diaires ?
   \item
Les fichiers que vous avez cr��s : o� sont-ils ? Comment sont-ils organis�s ? O� est le Makefile ?
   \item
Les outils ext�rieurs que vous avez utilis�s : la liste et le mode d'emploi pour les utiliser.
   \item
Les articles ou documentations annexes que vous avez r�dig�s, le cas �ch�ant, pendant votre stage : sont-ils signal�s dans votre rapport ?

   \item
Les figures que vous avez r�alis�es pendant votre stage : o� sont-elles ?

   \item
Les divers fichiers de simulation que vous avez cr��s pendant votre stage : o� sont-ils ? O� est le README permettant de les r�utiliser ?
   \end{itemize}
